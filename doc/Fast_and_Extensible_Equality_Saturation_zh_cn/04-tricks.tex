\section{用 \eclass 分析器拓展 \Egraphs}
% \section{Extending \Egraphs with \Eclass Analyses}
\label{sec:extensions}

% As discussed so far, \egraphs and equality saturation provide an efficient way
%   to implement a term rewriting system.
% Rebuilding enhances that efficiency, but the approach remains designed for
%   purely syntactic rewrites.
% However, program analysis and optimization typically require more than just
%   syntactic information.
正如迄今为止所讨论的,\egraphs 和 等式饱和 提供了一种高效的方法来实现一个项重写系统。
重建增强了效率,但该方法仍然仅限于语法重写。
然而,程序分析和优化通常需要的不仅仅是语法信息。
% Instead, transformations are \emph{computed} based on the input terms and also semantic facts
%   about that input term, e.g., constant value, free variables, nullability,
%   numerical sign, size in memory, and so on.
% The ``purely syntactic'' restriction has forced existing equality saturation
%   applications~\cite{eqsat, eqsat-llvm, herbie} to
%   resort to ad hoc passes over the \egraph
%   to implement analyses like constant folding.
% These ad hoc passes require manually manipulating the \egraph,
%   the complexity of which could prevent the implementation of more sophisticated
%   analyses.
%   % naturally expressed in the more conventional term rewriting setting,
%   % but must implemented as complicated \textit{ad hoc} passes over the \egraph.
相反,转换的 \emph{计算} 是基于输入 项 和该输入 项 的语义论据(semantic facts),
  例如,常量值,自由变量,空属性,数字符号,内存大小等。
“纯语法”限制迫使现有的 等式饱和 应用~\cite{eqsat, eqsat-llvm, herbie} 
  不得不求助于临时措施通过 \egraph 来实现常量折叠等分析。
这些临时措施需要手动操纵 \egraph ,其复杂性可能会妨碍实施更复杂的分析。


% We present a new technique called \textit{\eclass analysis},
%   which allows the concise
%   expression of a program analysis over the \egraph.
% An \eclass analysis resembles abstract interpretation
%   lifted to the \egraph level,
%   attaching \textit{analysis data} from a semilattice to each \eclass.
% The \egraph maintains and propagates this data as
%   \eclasses get merged and new \enodes are added.
% Analysis data can be used directly to modify the \egraph, to inform
%   how or if rewrites apply their right-hand sides, or to determine the cost of
%   terms during the extraction process.
我们提出了一种新技术,称为 \textit{\eclass 分析器(\eclass Analyses)},它允许在 \egraph 上简洁地表达程序分析。
一个 \eclass 分析器类似于提升到 \egraph 层面的抽象解释,
  从半格(semilattice)到每个\eclass 附加 \textit{分析数据}。% ?semilattice
在合并 \eclasses 和添加新的 \enodes 时,\egraph 会维护和传播这些数据。
分析数据可以直接用于修改 \egraph,以告知重写如何或是否应用其右右部,或在萃取过程中确定 term 的成本。

% \Eclass analyses provide a general mechanism to replace what previously
%   required ad hoc extensions that manually manipulate the \egraph.
% \Eclass analyses also fit within the equality saturation workflow,
%   so they can naturally cooperate with the equational reasoning provided by
%   rewrites.
% Moreover, an analysis lifted to the \egraph level automatically benefits from a
%   sort of ``partial-order reduction'' for free:
%   large numbers of similar programs may be analyzed for little additional cost
%   thanks to the \egraph's compact representation.
\eclass 分析器提供了一种通用机制,用来替换以前需要手动操纵 \egraph 的特殊扩展。
\eclass 分析器也适合 等式饱和 工作流程,
  因此它们可以自然地与重写提供的等式推理(equational reasoning)协作。
此外,在 \egraph 层面的分析会自动从一种“偏序归约(partial-order reduction)”中免费受益:
  借由 \egraph 的紧凑表示,大量类似程序可以以用很少的额外成本进行分析。

% This section provides a conceptual explanation of \eclass analyses as well
%   as dynamic and conditional rewrites that can use the analysis data.
% The following sections will provide concrete examples:
%   \autoref{sec:impl} discusses the \egg implementation and a complete example of a
%   partial evaluator for the lambda calculus;
%   \autoref{sec:case-studies} discusses how three published projects have used
%   \egg and its unique features (like \eclass analyses).
本节提供了 \eclass 分析器的概念解释以及可以使用分析数据的动态和条件重写。
接下来的章节将提供具体示例:
  \autoref{sec:impl} 讨论 \egg 实现和 lambda 演算中部分求值器的完整示例;
  \autoref{sec:case-studies} 讨论三个已发布项目如何使用
  \egg 及其独特特性(如 \eclass 分析器)。

% 原文注释 begin
% An \eclass analysis may,  modify the \egraph itself to inject new based on the computed data,
% In this way, the \eclass analysis can work with the rewrites as opposed

% \Egg offers many convenient ways to interact with the \egraph that are difficult
%   or impossible in other implementations.
% These tools give \egg the flexibility highlighted by the diverse case studies
%   in \autoref{sec:case-studies}
%   and the lambda calculus example in \autoref{sec:lambda}.

% Like the rest of \egg, these extensions are generic over the language and
%   rewrites that the user is working with.
% These are all novel in practice, as \egg is the first (to our knowledge)
%   general-purpose, reusable \egraph library.
% \Eclass analyses appear to be conceptually novel as well.

% \Chandra{this entire section is a bit light in details and doesn't
% always connect back to the goals of egg the right way. For example,
% without a concrete advantage of runners, it's not clear why it is useful.}
% 原文注释 end

% \subsection{\Eclass Analyses}
\subsection{\eclass 分析器}
\label{sec:analysis}

% An \eclass analysis defines a domain $D$ and associates a value $d_{c} \in D$ to
%   each \eclass $c$.
% The \eclass $c$ contains the associated data $d_{c}$,
%   i.e., given an \eclass $c$, one can get $d_{c}$ easily, but not vice-versa.
\eclass 分析器(\eclass Analyses)定义了一个域 $D$ 并将一个值 $d_{c} \in D$ 与每个 \eclass $c$ 关联。
\eclass $c$ 包含关联数据 $d_{c}$,
  即,给定一个 \eclass $c$,可以很容易地获得 $d_{c}$,但反过来不行。

% The interface of an \eclass analysis is as follows,
%   where $G$ refers to the \egraph,
%   and $n$ and $c$ refer to \enodes and \eclasses within $G$:
\eclass 分析器的接口如下,
  其中 $G$ 指的是 \egraph,
  $n$ 和 $c$ 指的是 $G$ 中的 \enodes 和 \eclasses :

% 原文注释 begin
% We will use the following metavariables and syntax in this section:

% \begin{center}
%   $G$: \egraph, $c$: \eclass, $n$: \enode,
%   $d_{c}$: the analysis data associated with \eclass $c$
% \end{center}
% 原文注释 end

% 本段翻译未仔细校对
\vspace{1em}
\begin{tabular}{lp{0.7\linewidth}}
  $\textsf{make}(n) \to d_{c}$ &
    % When a new \enode $n$ is added to $G$ into a new, singleton \eclass $c$,
    % construct a new value $d_{c} \in D$ to be associated with $n$'s new \eclass,
    % typically by accessing the associated data of $n$'s children.
    当一个新的 \enode $n$ 被添加到 $G$ 并且形成一个新的,单一的 \eclass $c$ 时,
    构造一个新值 $d_{c} \in D$ 与 $n$ 的新 \eclass 关联,
    通常通过访问 $n$ 的子节点的关联数据。
  \\
  $\textsf{join}(d_{c_1}, d_{c_2}) \to d_{c}$ &
    % When \eclasses $c_{1}, c_{2}$ are being merged into $c$,
    % join $d_{c_1}, d_{c_2}$ into a new value $d_{c}$ to be associated with the
    % new \eclass $c$.
    当 \eclasses $c_{1}, c_{2}$ 被合并成 $c$ 时,将 $d_{c_1}, d_{c_2}$ 合并成一个新值 $d_{c}$ 与新 \eclass $c$ 关联。
  \\
  $\textsf{modify}(c) \to c'$ &
    % Optionally modify the \eclass $c$ based on $d_{c}$, typically by adding an
    %   \enode to $c$.
    % Modify should be idempotent if no other changes occur to the \eclass, i.e.,
    %   $\textsf{modify}(\textsf{modify}(c)) = \textsf{modify}(c)$
    % % Returning $c$ unmodified ($c = c'$) suffices in many cases, and is the
    % % default implementation for analyses in \egg.
    根据 $d_{c}$ 可选地修改 \eclass $c$,通常是添加一个 \enode 到 $c$。
    如果 \eclass 没有其他变化,修改应该幂等,即 
      $\textsf{modify}(\textsf{modify}(c)) = \textsf{modify}(c)$
\end{tabular}
\vspace{1em}

% The domain $D$ together with the \textsf{join} operation should form a join-semilattice.
% The semilattice perspective is useful for defining the \textit{analysis invariant}
%   (where $\wedge$ is the \textsf{join} operation):
域 $D$ 与 \textsf{join} 操作一起应该形成一个联并半格(join-semilattice)。 % ?
半格视角对于定义 \textit{分析不变量} 是很有用的。 % ?
  (其中 $\wedge$是textsf{join}操作)。
\[
  \forall c \in G.\quad
  d_{c} = \bigwedge_{n \in c} \textsf{make}(n)
  \quad \text{and} \quad
  \textsf{modify}(c) = c
\]

% 本段翻译未仔细校对
% The first part of the analysis invariant states that the data associated with
%   each \eclass must be the \textsf{join} of the \textsf{make} for every \enode
%   in that \eclass.
% Since $D$ is a join-semilattice, this means that
%   $\forall c, \forall n \in c, d_{c} \geq \textsf{make}(n) $.
% The motivation for the second part is more subtle.
% Since the analysis can modify an \eclass through the \textsf{modify} method,
%   the analysis invariant asserts that these modifications are driven to a fixed
%   point.
% When the analysis invariant holds, a client looking at the analysis data can be
%   assured that the analysis is ``stable'' in the sense that
%   recomputing \textsf{make}, \textsf{join}, and \textsf{modify} will not
%   modify the \egraph or any analysis data.
第一部分的分析不变性声明,
  每个 \eclass 的关联的数据必须是每个 \enode 的 \textsf{make} 的\textsf{join}。
由于 $D$ 是 联并半格(join-semilattice) ,这意味着
  $\forall c, \forall n \in c, d_{c} \geq \textsf{make}(n) $。
第二部分的动机更为微妙。
由于分析可以通过 \textsf{modify} 方法修改一个\eclass,
  分析不变性声明这些修改被驱动到一个固定点。
当分析不变性成立时,客户端看到的分析数据可以确保分析在重新计算
  \textsf{make},\textsf{join},\textsf{modify} 不会修改 \egraph 或任何分析数据的意义下是“稳定”的。

% \subsubsection{Maintaining the Analysis Invariant}
\subsubsection{保持分析不变量(Analysis Invariant)}

% 因模板问题,代码注释暂不翻译
\begin{figure}
  \begin{minipage}[t]{0.47\linewidth}
    \begin{lstlisting}[gobble=4, numbers=left, numberstyle=\color{black}, basicstyle=\scriptsize\ttfamily\color{black!40}, escapechar=|]
    def add(enode):
      enode = self.canonicalize(enode)
      if enode in self.hashcons:
        return self.hashcons[enode]
      else:
        eclass = self.new_singleton_eclass(enode)
        for child_eclass in enode.children:
          child_eclass.parents.add(enode, eclass)
        self.hashcons[enode] = eclass
        |\color{black}\label{line:add1} eclass.data = analysis.make(enode)|
        |\color{black}\label{line:add2} analysis.modify(eclass)|
        return eclass

    def merge(eclass1, eclass2)
      union = self.union_find.union(eclass1, eclass2)
      if not union.was_already_unioned:
        |\color{black}\label{line:merge1}d1, d2 = eclass1.data, eclass2.data|
        |\color{black}\label{line:merge2}union.eclass.data = analysis.join(d1, d2)|
        self.worklist.add(union.eclass)
      return union.eclass
    \end{lstlisting}
  \end{minipage}
  \hfill
  \begin{minipage}[t]{0.47\linewidth}
    \begin{lstlisting}[gobble=4, numbers=left, firstnumber=21, numberstyle=\color{black}, basicstyle=\scriptsize\ttfamily\color{black!40}, escapechar=|]
    def repair(eclass):
      for (p_node, p_eclass) in eclass.parents:
        self.hashcons.remove(p_node)
        p_node = self.canonicalize(p_node)
        self.hashcons[p_node] = self.find(p_eclass)

      new_parents = {}
      for (p_node, p_eclass) in eclass.parents:
        p_node = self.canonicalize(p_node)
        if p_node in new_parents:
          self.union(p_eclass, new_parents[p_node])
        new_parents[p_node] = self.find(p_eclass)
      eclass.parents = new_parents
    \end{lstlisting}
    \vspace{-3mm}
    \begin{lstlisting}[gobble=4, numbers=left, firstnumber=34, basicstyle=\scriptsize\ttfamily, escapechar=|]

      # 任何对 eclass 的修改
      # 都会添加到 worklist 中
      |\label{line:repair1}|analysis.modify(eclass)
      for (p_node, p_eclass) in eclass.parents:
        new_data = analysis.join(
          p_eclass.data,
          analysis.make(p_node))
        if new_data != p_eclass.data:
          p_eclass.data = new_data
          |\label{line:repair2}|self.worklist.add(p_eclass)
    \end{lstlisting}
  \end{minipage}
  \caption{
    % The pseudocode for maintaining the \eclass analysis invariant is largely
    %   similar to how rebuilding maintains congruence closure
    %   (\autoref{sec:rebuilding}).
    % Only lines \ref{line:add1}--\ref{line:add2},
    %   \ref{line:merge1}--\ref{line:merge2},
    %   and \ref{line:repair1}--\ref{line:repair2} are added.
    % Grayed out or missing code is unchanged from \autoref{fig:rebuild-code}.
    维护 \eclass 分析器不变量的伪代码大体上与重建维持同余闭包的方式相似(\autoref{sec:rebuilding})。
      只添加了行 \ref{line:add1}--\ref{line:add2},
      \ref{line:merge1}--\ref{line:merge2},
      和 \ref{line:repair1}--\ref{line:repair2} 。
    灰色或未列出的代码与 \autoref{fig:rebuild-code} 相同。
    % 【代码注释翻译】
    %   # any mutations modify makes to eclass
    %   # will add to the worklist
    %   任何对 eclass 的修改
    %   都会添加到 worklist 中
  }
  \label{fig:rebuild-analysis}
\end{figure}

% 本段翻译未仔细校对
% We extend the rebuilding procedure from \autoref{sec:rebuilding} to restore the
%   analysis invariant as well as the congruence invariant.
% \autoref{fig:rebuild-analysis} shows the necessary modifications to the
%   rebuilding code from \autoref{fig:rebuild-code}.
我们将重建过程从 \autoref{sec:rebuilding} 扩展到恢复分析不变性和同余不变性。
\autoref{fig:rebuild-analysis}显示了从\autoref{fig:rebuild-code}修改重建代码所需的修改。


% 本段翻译未仔细校对
% Adding \enodes and merging \eclasses risk breaking the analysis invariant in
%   different ways.
% Adding \enodes is the simpler case; lines \ref{line:add1}--\ref{line:add2}
%   restore the invariant for the newly created, singleton \eclass that holds the
%   new \enode.
% When merging \enodes, the first concern is maintaining the semilattice portion of the
%   analysis invariant.
% Since \textsf{join} forms a semilattice over the domain $D$ of the analysis
%   data, the order in which the joins occur does not matter.
% Therefore, line \ref{line:merge2} suffices to update the analysis data of the
%   merged \eclass.
添加 \enodes 和合并 \eclasses 都有可能以不同的方式打破分析不变性。
添加\enodes 是更简单的情况;
  行 \ref{line:add1}--\ref{line:add2} 恢复了新创建的,单个 \eclass 的不变性。%?singleton
合并\enodes 时,第一个关注点是维护分析不变性的半阶部分。
由于\textsf{join}在分析数据的域D上形成半阶,因此合并的顺序并不重要。
因此,线\ref{line:merge2}足以更新合并的\eclass 的分析数据。

% 本段翻译未仔细校对
% Since $\textsf{make}(n)$ creates analysis data by looking at the data of $n$'s,
%   children, merging \eclasses can violate the analysis invariant in the same way
%   it can violate the congruence invariant.
% The solution is to use the same worklist mechanism introduced in
%   \autoref{sec:rebuilding}.
% Lines \ref{line:repair1}--\ref{line:repair2} of the \texttt{repair} method
%   (which \texttt{rebuild} on each element of the worklist)
%   re-\textsf{make} and \textsf{merge} the analysis data of the parent of any
%   recently merged \eclasses.
% The new \texttt{repair} method also calls \textsf{modify} once, which suffices
%   due to its idempotence.
% In the pseudocode, \textsf{modify} is reframed as a mutating method for clarity.
由于 $\textsf{make}(n)$ 通过查看 $n$ 的子元素创建分析数据,
  因此合并 \eclasses 可能会在同一方式中违反分析不变性。
解决方案是使用在 \autoref{sec:rebuilding} 中引入的相同工作列表机制。
\texttt{repair} 方法 (它在工作列表的每个元素上\texttt{rebuild}) 
  的行 \ref{line:repair1}--\ref{line:repair2} 
  重新 \textsf{make}和 \textsf{merge} 最近合并的 \eclasses 父元素的分析数据。
新的 \texttt{repair} 方法还调用 \textsf{modify} 一次,这足以满足其幂等性(idempotence)。
在伪代码中,为了更明晰,\textsf{modify} 被重构为一种可变方法。


% 本段翻译未仔细校对
% \Egg's implementation of \eclass analyses assumes that the analysis domain $D$
%   is indeed a semilattice and that \textsf{modify} is idempotent.
% Without these properties, \egg may fail to restore the analysis invariant on
%   \texttt{rebuild}, or it may not terminate.
\Egg 实现的 \eclass 分析器假设分析域 $D$ 确实是半格,
  并且 \textsf{modify} 具有幂等性。
如果没有这些性质,\egg 可能无法在 \texttt{rebuild} 上恢复分析不变量,或者它可能永不终止。

% % \subsubsection{Modifying the \Egraph from an \Eclass Analysis}
% \subsubsection{Example: Constant Folding}
\subsubsection{示例: 常量折叠(Constant folding)}


% 本段翻译未仔细校对
% The data produced by \eclass analyses can be
%   usefully consumed by other components of an equality saturation system
%   (see \autoref{sec:rewrites}),
%   but \eclass analyses can be useful on their own thanks to the
%   \textsf{modify} hook.
% Typical \textsf{modify} hooks will either do nothing, check some invariant about
%   the \eclasses being merged, or add an \enode to that \eclass
%   (using the regular \texttt{add} and \texttt{merge} methods of the \egraph).
\eclass 分析器产生的数据可以被其他 等式饱和 系统组件有效地使用(参见 \autoref{sec:rewrites}),
但是由于 \textsf{modify} 钩子的存在,\eclass 分析器可以独立使用
典型的 \textsf{modify} 钩子 会什么事都不做,检查要合并的 \eclasses 的某些不变量,
或者将一个 \enode 添加到该 \eclass 中(使用 \egraph 的常规 \texttt{add} 和 \texttt{merge} 方法)。

% 本段翻译未仔细校对
% As mentioned above, other equality saturation implementations have implemented
%   constant folding as custom, ad hoc passes over the \egraph.
% We can formulate constant folding as an \eclass analysis that highlights the
%   parallels with abstract interpretation.
% Let the domain $D = \texttt{Option<Constant>}$, and let the \texttt{join}
%   operation be the ``\texttt{or}'' operation of the \texttt{Option} type:
如上所述,其他 等式饱和 实现已经将常量折叠实现为自定义的,
  临时通过 \egraph 的步骤。 %?
我们可以将常量折叠表示为一个 \eclass 分析器,突出了它与抽象解释之间的相似之处。
设域 $D = \texttt{Option<Constant>}$,
  设 \texttt{join} 操作是 \texttt{Option} 类型上的 ``\texttt{or}'' 操作:  
\\
\begin{minipage}{\linewidth}
\begin{lstlisting}[language=Rust, basicstyle=\ttfamily\footnotesize, xleftmargin=35mm]
match (a, b) {
  (None,    None   ) => None,
  (Some(x), None   ) => Some(x),
  (None,    Some(y)) => Some(y),
  (Some(x), Some(y)) => { assert!(x == y); Some(x) }
}
\end{lstlisting}
\end{minipage}
\\
% 本段翻译未仔细校对
% Note how \textsf{join} can also aid in debugging by checking properties about
%   values that are unified in the \egraph;
%   in this case we assert that all terms represented in an \eclass should have
%   the same constant value.
% The \textsf{make} operation serves as the abstraction function, returning the
%   constant value of an \enode if it can be computed from the constant values
%   associated with its children \eclasses.
% The \textsf{modify} operation serves as a concretization function in this
%   setting.
% If $d_{c}$ is a constant value, then $\textsf{modify}(c)$ would add
%   $\gamma(d_{c}) = n$ to $c$, where $\gamma$ concretizes the constant value into
%   a childless \enode.
注意如何 \textsf{join} 也可以通过检查在 \egraph 中统一的值的属性来辅助调试;
  在这种情况下,我们断言在 \eclass 中表示的所有项应该具有相同的常量值。
\textsf{make} 操作充当抽象函数,
  如果它可以从与其子\eclasses 相关联的常量值计算出来,
  返回 \enode 的常量值。
\textsf{modify} 操作在这个设置中充当具体化函数。 %  concretization?
如果 $d_{c}$ 是一个常量值,那么 $\textsf{modify}(c)$ 就会将 
  $\gamma(d_{c}) = n$ 添加到 $c$ 中,
  其中 $\gamma$ 将常量值具体化(concretizes)为没有子节点的 \enode。

% 本段翻译未仔细校对
% Constant folding is an admittedly simple analysis, but one that did not formerly
%   fit within the equality saturation framework.
% \Eclass analyses support more complicated analyses in a general way, as
%   discussed in later sections on the \egg implementation and case studies
%   (Sections \ref{sec:impl} and \ref{sec:case-studies}).
常量折叠是一种明显简单的分析,但之前并不适合 等式饱和 框架。
\Eclass analyses 在通用的方式中支持更复杂的分析,
  如后面关于 \egg 实现和案例研究的章节所讨论的
  (\ref{sec:impl} 和 \ref{sec:case-studies})。

% \subsection{Conditional and Dynamic Rewrites}
\subsection{条件和动态重写}
\label{sec:rewrites}

% 此段翻译未仔细校对
% In equality saturation applications, most of the rewrites are purely
%   syntactic.
% In some cases, additional data may be needed to determine if or how to perform
%   the rewrite.
% For example, the rewrite $x / x \to 1$ is only valid if $x \neq 0$.
% A more complex rewrite may need to compute the right-hand side dynamically based
%   on an analysis fact from the left-hand side.
在 等式饱和 应用中,大多数重写都是纯粹的语法形式。
在某些情况下,可能需要额外的数据来确定是否执行重写或如何执行重写。
例如,$x / x \to 1$ 重写仅在 $x \neq 0$ 时有效。
更复杂的重写可能需要根据左侧的分析事实动态计算右侧。

% 此段翻译未仔细校对
% The right-hand side of a rewrite can be generalized to a function
%   \textsf{apply} that takes a substitution and an \eclass generated from
%   e-matching the left-hand side, and produces a term to be added to the \egraph
%   and unified with the matched \eclass.
% For a purely syntactic rewrite, the \textsf{apply} function need not inspect the
%   matched \eclass in any way; it would simply apply
%   the substitution to the right-hand pattern to produce a new term.
重写的右侧可以概括为函数 \textsf{apply},该函数采用替代和通过 e-matching 左侧得到的 \eclass,并产生一个术语,该术语将被添加到 \egraph 中并与匹配的 \eclass 统一。
对于纯粹的语法重写,\textsf{apply} 函数无需以任何方式检查匹配的 \eclass;它只需将替代应用于右侧模式以产生新术语即可。

% 此段翻译未仔细校对
% \Eclass analyses greatly increase the utility of this generalized form of
%   rewriting.
% The \textsf{apply} function can look at the analysis data for the matched
%   \eclass or any of the \eclasses in the substitution to determine if or how to
%   construct the right-hand side term.
% These kinds of rewrites can broken down further into two categories:
\eclass 分析器大大增加了这种重写的通用形式的效用。
\textsf{apply} 函数可以查看匹配的 \eclass 的分析数据或替代中的任何 \eclasses,以确定如何构造右侧术语。这些重写可以进一步分为两类:
\begin{itemize}
  % \item \textit{Conditional} rewrites like $x / x \to 1$ that are purely
  % syntactic but whose validity depends on checking some analysis data;
  % \item \textit{Dynamic} rewrites that compute the right-hand side based on
  % analysis data.
  \item \textit{条件} 重写,如 $x / x \to 1$,它们是纯粹的语法形式,
    但其有效性取决于检查一些分析数据。
  \item \textit{动态} 重写,它们根据分析数据计算右侧。
\end{itemize}

% 此段翻译未仔细校对
% Conditional rewrites are a subset of the more general dynamic rewrites.
% Our \egg implementation supports both.
% The example in \autoref{sec:impl} and case studies in \autoref{sec:case-studies}
%   heavily use generalized rewrites, as it is typically the most convenient way
%   to incorporate domain knowledge into the equality saturation
%   framework.
条件重写是更一般动态重写的子集。我们的 \egg 实现支持两者。在 \autoref{sec:impl} 中的示例和在 \autoref{sec:case-studies} 中的案例研究中大量使用了广义重写,因为这通常是将领域知识纳入 等式饱和 框架的最方便的方法。

% \subsection{Extraction}
\subsection{萃取(Extraction)}
\label{sec:tricks-extraction}

% 此段翻译未仔细校对
% Equality saturation typically ends with an extraction phase that selects an
%   optimal represented term from an \eclass according to some cost function.
% In many domains \cite{herbie, szalinski}, AST size
%   (sometimes weighted differently for different operators) suffices as a simple,
%   local cost function.
% We say a cost function $k$ is local when the cost of a term $f(a_{1}, ...)$ can be
%   computed from the function symbol $f$ and the costs of the children.
% With such cost functions, extracting an optimal term can be efficiently done
%   with a fixed-point traversal over the \egraph that selects the minimum cost
%   \enode from each \eclass \cite{herbie}.
等式饱和 通常以提取阶段结束,该阶段根据某些成本函数从 \eclass 中选择最优表示术语。
在许多领域 \cite{herbie, szalinski} 中,
  AST 大小(有时对不同运算符的权重不同)足以作为简单的局部成本函数。 % ?AST
当我们可以从函数符号 $f$ 和子结点的成本计算出术语 $f(a_{1}, ...)$ 的成本时,
  称成本函数 $k$ 为局部的。
使用这样的成本函数,可以使用固定点遍历 \egraph 来高效地提取最优术语,该遍历从每个 \eclass 中选择最小成本的 \enode \cite{herbie}。

% 此段翻译未仔细校对
% Extraction can be formulated as an \eclass analysis when the cost function
%   is local.
% The analysis data is a tuple $(n, k(n))$ where $n$ is the cheapest \enode
%   in that \eclass and $k(n)$ its cost.
% The $\textsf{make}(n)$ operation calculates the cost $k(n)$ based on
%   the analysis data (which contain the minimum costs) of $n$'s children.
% The \textsf{merge} operation simply takes the tuple with lower cost.
% The semilattice portion of the analysis invariant then guarantees that the
%   analysis data will contain the lowest-cost \enode in each class.
% Extract can then proceed recursively;
%   if the analysis data for \eclass $c$ gives $f(c_{1}, c_{2}, ...)$ as the optimal \enode,
%   the optimal term represented in $c$ is
%   $\textsf{extract}(c) = f( \textsf{extract}(c_{1}), \textsf{extract}(c_{2}), ... )$.
% % The optimal term represented in an \eclass can then be built recursively,
% %   starting with the optimal \enode from the analysis data.
% % Extraction can be completed by starting from the desired \eclass and
% %   building the term recursively based on the \enode from the analysis data.
% This not only further demonstrates the generality of \eclass analyses, but also
%   provides the ability to do extraction ``on the fly''; conditional and dynamic
%   rewrites can determine their behavior based on the cheapest term in an \eclass.
萃取可以在成本函数是局部的情况下被制定(formulated)为 \eclass 分析器。 %?formulated
分析数据是一个元组 $(n, k(n))$,其中 $n$ 是该 \eclass 中最便宜的 \enode,
  $k(n)$ 是它的花费。
$\textsf{make}(n)$ 操作基于 $n$ 的子节点的分析数据(其中包含最小花费)计算 $k(n)$。
\textsf{merge} 操作只需取较低花费的元组。
分析不变量的半格部分保证了分析数据将包含每个类中最低花费的 \enode。
提取可以递归进行;
  如果 \eclass $c$ 的分析数据给出 $f(c_{1}, c_{2}, ...)$ 作为最佳 \enode,
  则 $c$ 表示的最佳项是  
  $\textsf{extract}(c) = f( \textsf{extract}(c_{1}), \textsf{extract}(c_{2}), ... )$。
这不仅进一步证明了 \eclass 分析器的普遍性,还提供了“即时”提取的能力;
  条件和动态重写可以根据分析数据来确定其行为。

% 此段翻译未仔细校对
% Extraction (whether done as a separate pass or an \eclass analysis) can also
%   benefit from the analysis data.
% Typically, a local cost function can only look at the function symbol of the
%   \enode $n$ and the costs of $n$'s children.
% When an \eclass analysis is attached to the \egraph, however, a cost function
%   may observe the data associated with $n$'s \eclass, as well as the data
%   associated with $n$'s children.
% This allows a cost function to depend on computed facts rather that just purely
%   syntactic information.
% In other words, the cost of an operator may differ based on its inputs.
% \autoref{sec:spores} provides a motivating case study wherein an \eclass
%   analysis computes the size and shape of tensors, and this size information
%   informs the cost function.
萃取(无论是作为单独的通过还是作为 \eclass 分析器)也可以从分析数据中受益。
通常,局部成本函数只能查看 \enode $n$ 的函数符号和 $n$ 的子节点的成本。
但是,当 \eclass 分析器附加到 \egraph 时,
  成本函数可以观察与 $n$ 的 \eclass 关联的数据,
  以及与 $n$ 的子节点关联的数据。
这允许成本函数依赖于计算出的事实而不仅仅是纯语法信息。
换句话说,运算符的成本可能因其输入而不同。
\autoref{sec:spores} 提供了一个激励性的案例研究(motivating case study ),% ? motivating case study 
  其中 \eclass 分析器计算张量的大小和形状,并且这些大小信息会影响成本函数。

%%% Local Variables:
%%% TeX-master: "egg"
%%% End:
