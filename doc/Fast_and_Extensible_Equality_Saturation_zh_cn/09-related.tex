\section{相关工作}
% \section{Related Work}
\label{sec:related}
% 草草翻译完成

% \paragraph{Term Rewriting}
\paragraph{Term 重写}

% 此段翻译未仔细校对 :-( 
% Term rewriting~\cite{rewritesystems} has been used widely to facilitate
% equational reasoning for program
% optimizations~\cite{DBLP:conf/scitools/BoyleHW96,
%   DBLP:journals/toplas/BrandHKO02, DBLP:conf/icfp/VisserBT98}. A term rewriting
% system applies a database of semantics preserving rewrites or axioms to an input
% expression to get a new expression, which may, according to some cost function,
% be more profitable compared to the input. Rewrites are typically symbolic and
% have a left hand side and a right hand side. To apply a rewrite to an
% expression, a rewrite system implements pattern matching---if the left hand side
% of a rewrite rule matches with the input expression, the system computes a
% substitution which is then applied to the right-hand side of the rewrite rule.
% Upon applying a rewrite rule, a rewrite system typically replaces the old
% expression by the new expression. This can lead to the \textit{phase ordering}
% problem--- it makes it impossible to apply a rewrite to the old expression in
% the future which could have led to a more optimal result. 
Term 重写~\cite{rewritesystems} 已被广泛用于促进程序优化~\cite{DBLP:conf/scitools/BoyleHW96,
  DBLP:journals/toplas/BrandHKO02, DBLP:conf/icfp/VisserBT98}的等式推理中。
Term 重写系统将保留语义的重写或公理的数据库应用于输入表达式,
  以得到一个新的表达式,根据一些成本函数,这个表达式与输入式相比可能更有利。
  重写通常是符号化的,有一个左式和一个右式。
  为了对一个表达式进行重写,重写系统实现了模式匹配
  ——如果重写规则的左侧与输入表达式相匹配,系统就会计算出一个替换,然后应用到重写规则的右侧。
  在应用重写规则时,重写系统通常会用新的表达式替换旧的表达式。
  这可能会导致 \textit{阶段排序(phase ordering)} 的问题
  —— 它使得在未来不可能对旧的表达式进行重写,而这可能会导致更优的结果。


% \paragraph{\Egraphs and E-matching}
\paragraph{\Egraphs 和 E-matching}

% 此段翻译未仔细校对 :-( 
% \Egraph were originally proposed several decades ago as an
% efficient data structure for maintaining congruence
% closure~\cite{nelson-oppen-78, kozen-stoc77, nelson}.
% \Egraphs continue to be a critical component in successful SMT
% solvers where they are used for combining satisfiability theories by sharing equality
% information~\cite{z3}.
%  A key difference between past implementations of \egraphs and
%   \egg's \egraph is our novel rebuilding algorithm that maintains
%   invariants only at certain critical points~(\autoref{sec:rebuild}).
% This makes \egg more efficient for the purpose of equality saturation.
%   \egg implements the pattern compilation strategy introduced by de Moura et al.~\cite{ematching}
%   that is used in state of the art theorem provers~\cite{z3}.
\Egraph 最初是几十年前提出的一种高效的数据结构,
  用于维护同余闭包~\cite{nelson-oppen-78, kozen-stoc77, nelson}。
\Egraphs 仍然是成功的 SMT 求解器中的关键组成部分,
  它们被用于通过共享等价关系结合可满足性理论~\cite{z3}。 % ?
过去 \egraph 实现和 \egg 的 \egraph 的主要区别在于我们新颖的重建算法——
  只在某些关键点维护不变性~(\autoref{sec:rebuild})。
这使得 \egg 更适合等价饱和的目的。
\egg 实现了 de Moura 等人提出的模式编译策略~\cite{ematching},
  该策略在最先进的定理证明器~\cite{z3}中使用。
% Some provers~\cite{z3, simplify} propose optimizations like mod-time,
% pattern-element and inverted-path-index to find new terms and relevant patterns
% for matching, and avoid redundant matches. So far, we have found \egg to be
% faster than several prior \egraph implementations even without
% these optimizations. They are,
% however, compatible with \egg's design and could be explored in the future. Another
% key difference is \egg's powerful \eclass analysis abstraction and flexible
% interface. They empower the programmer to easily leverage \egraphs for problems
% involving complex semantic reasoning.
一些证明器~\cite{z3, simplify} 提出了像 
  mod-time,pattern-element 和 inverted-path-index 这样的优化,
  用于找到新的术语和相关模式进行匹配,避免重复匹配。
  到目前为止,我们发现 \egg 比之前的几个 \egraph 实现更快,即使没有这些优化。
然而,它们与 \egg 的设计兼容,可以在未来进行探索。
  另一个关键差异是 \egg 强大的 \eclass 分析抽象和灵活的界面。
  它们使程序员能够轻松地利用 \egraph 解决涉及复杂语义推理的问题。

% \paragraph{Congruence Closure}
\paragraph{同余闭包}

% 此段翻译未仔细校对 :-( 
% Our rebuilding algorithm is similar to the congruence closure algorithm
%   presented by \citet{downey-cse}.
% The contribution of rebuilding is not \emph{how} it restores the \egraph invariants
%   but \emph{when};
%   it gives the client the ability to specialize invariant restoration to a
%   particular workload like equality saturation.
% Their algorithm also features a worklist of merges to be processed further,
%   but it is offline, i.e.,
%   the algorithm processes a given set of equalities and outputs the set of
%   equalities closed over congruence.
% Rebuilding is adapted to the online e-graph (and equality saturation) setting,
%   where rewrites frequently examine the current set of equalities and assert new ones.
% Rebuilding additionally propagates e-class analysis facts (\autoref{sec:analysis}).
% Despite these differences,
%   the core algorithms algorithms are similar enough that theoretical results on
%   offline performance characteristics \cite{downey-cse} apply to both.
% We do not provide theoretical analysis of rebuilding for the online setting;
%   it is likely highly workload dependent.
我们的重建算法类似于 \citet{downey-cse} 提出的同余闭包(congruence closure)算法。
重建的贡献不在于它如何恢复 \egraph 不变性,而在于它什么时候恢复;
  客户端有能力将不变性恢复专门化为特定工作负载,如 等式饱和。
它的算法也具有要进一步处理的合并工作列表
  ,但它是离线的,即算法处理给定的等价关系并输出经过同构闭包的等价关系集。
重建适用于在线 e-graph(和等式饱和)设置,
  在这种设置中重写经常检查当前的等价关系并断言新的关系。
重建还传播 e-class 分析事实(\autoref{sec:analysis})。
尽管存在这些差异,但核心算法相似,离线性能特征 \cite{downey-cse} 适用于两者。
我们没有为在线设置提供重建的理论分析;它可能与工作负载高度相关。

% \paragraph{Superoptimization and Equality Saturation}
\paragraph{超优化(Superoptimization)~和~ 等式饱和}

% 此段翻译未仔细校对 :-( 
% The Denali~\cite{denali} superoptimizer first demonstrated how to use \egraphs
% for optimized code generation as an alternative to hand-optimized machine code
% and prior exhaustive approaches~\cite{massalin}, both of which were less
% scalable. The inputs to Denali are programs in a C-like language from which it
% produces assembly programs. Denali supported three types of
% rewrites---arithmetic, architectural, and program-specific. After applying these
% rewrites till saturation, it used architectural description of the hardware to
% generate constraints that were solved using a SAT solver to output a
% near-optimal program. While Denali's approach was a significant improvement over
% prior work, it was intended to be used on straight line code only and therefore,
% did not apply to large real programs.
Denali~\cite{denali} 超优化器首先展示了如何使用 \egraphs 进行优化代码生成,
作为手动优化机器代码和先前的详尽方法~\cite{massalin}的替代方案,两者都不易扩展。 
Denali 的输入是类 C 语言中的程序,它会生成汇编程序。 
Denali 支持三种重写——算术、架构和特定于程序。
在应用这些重写直到饱和之后,它使用硬件描述生成约束,使用 SAT 求解器解决这些约束以输出近似最优程序。
虽然 Denali 的方法比先前的工作有了重大改进,但它只适用于直线代码,因此不适用于大型实际程序。

% 此段翻译未仔细校对 :-( 
% Equality saturation~\cite{eqsat, eqsat-llvm} developed a compiler optimization
% phase that works for complex language constructs like loops and conditionals.
% The first equality saturation paper used an intermediate representation called
% Program Expression Graphs (PEGs) to encode loops and conditionals. PEGs have
% specialized nodes that can represent infinite sequences, which allows them to
% represent loops. It uses a global profitability heuristic for extraction which
% is implemented using a pseudo-boolean solver. Recently, \cite{yogo-pldi20} uses
% PEGs for code search.
% \egg can support PEGs as a user-defined language,
% and thus their technique could be ported.
等式饱和 ~\cite{eqsat, eqsat-llvm} 开发了编译优化选择算法,
适用于循环和条件等复杂语言结构。
第一篇关于等价饱和的论文使用中间表示,称为程序表达图(PEGs)来编码循环和条件。
PEGs 有专门的节点,可以表示无限序列,这样就可以表示循环。
它使用全局可盈利性启发式来进行提取,使用伪布尔求解器实现。
最近,\cite{yogo-pldi20} 使用 PEGs 进行代码搜索(code search)。
\egg 可以支持 PEGs 作为用户定义语言,因此它们的技术可以移植。


%PEGs could be implemented in \egg.

%%% Local Variables:
%%% TeX-master: "egg"
%%% End:
