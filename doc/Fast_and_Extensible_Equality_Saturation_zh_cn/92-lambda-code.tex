% 翻译完成
\begin{figure}
\begin{minipage}[t]{0.49\linewidth}
  \begin{lstlisting}[language=Rust, basicstyle=\tiny\ttfamily, numbers=left]
type EGraph = egg::EGraph<Lambda, LambdaAnalysis>;
struct LambdaAnalysis;
struct FC {
  free: HashSet<Id>,    // 我们的分析数据存储自由变量
  constant: Option<Lambda>, // 以及常量值(如果有)
}

// 帮助函数,用于制作模式元变量(pattern meta-variables)
fn var(s: &str) -> Var { s.parse().unwrap() }

impl Analysis<Lambda> for LambdaAnalysis {
  type Data = FC; // 将 FC 附加到每个 eclass
  // merge 通过合并到 “to” 实现半格联并 (semilattice join)
  // 如果 “to” 数据被修改,则返回 true
  fn merge(&self, to: &mut FC, from: FC) -> bool {
    let before_len = to.free.len();
    // union the free variables 联并自由变量
    to.free.extend(from.free.iter().copied());
    if to.constant.is_none() && from.constant.is_some() {
      to.constant = from.constant;
      true
    } else {
      before_len != to.free.len()
    }
  }

  fn make(egraph: &EGraph, enode: &Lambda) -> FC {
    let f = |i: &Id| egraph[*i].data.free.iter().copied();
    let mut free = HashSet::default();
    match enode {
      Use(v) => { free.insert(*v); }
      Let([v, a, b]) => {
        free.extend(f(b)); free.remove(v); free.extend(f(a));
      }
      Lambda([v, b]) | Fix([v, b]) => {
        free.extend(f(b)); free.remove(v);
      }
      _ => enode.for_each_child(
             |c| free.extend(&egraph[c].data.free)),
    }
    FC { free: free, constant: eval(egraph, enode) }
  }

  fn modify(egraph: &mut EGraph, id: Id) {
    if let Some(c) = egraph[id].data.constant.clone() {
      let const_id = egraph.add(c);
      egraph.union(id, const_id);
    }
  }
}\end{lstlisting}
\end{minipage}
\hfill
\begin{minipage}[t]{0.46\linewidth}
  \begin{lstlisting}[language=Rust, basicstyle=\tiny\ttfamily, escapechar=@, numbers=left, firstnumber=51]
// 如果子元素有常量,评估 enode
// Rust的 `?` 提取一个 Option,如果是 None,则提前返回。
fn eval(eg: &EGraph, enode: &Lambda) -> Option<Lambda> {
  let c = |i: &Id| eg[*i].data.constant.clone();
  match enode {
    Num(_) | Bool(_) => Some(enode.clone()),
    Add([x, y]) => Some(Num(c(x)? + c(y)?)),
    Eq([x, y]) => Some(Bool(c(x)? == c(y)?)),
    _ => None,
  }
}

// 这种类型的函数可以作为重写的条件
trait ConditionFn = Fn(&mut EGraph, Id, &Subst) -> bool;

// 以下两个函数返回正确签名的闭包,
// 它可用作 @\autoref{fig:lambda-rules}@ 中的条件
fn is_not_same_var(v1: Var, v2: Var) -> impl ConditionFn {
    |eg, _, subst| eg.find(subst[v1]) != eg.find(subst[v2])
}
fn is_const(v: Var) -> impl ConditionFn {
     // check the LambdaAnalysis data
    |eg, _, subst| eg[subst[v]].data.constant.is_some()
}

struct CaptureAvoid {
  fresh: Var, v2: Var, e: Var,
  if_not_free: Pattern<Lambda>, if_free: Pattern<Lambda>,
}

impl Applier<Lambda, LambdaAnalysis> for CaptureAvoid {
  // 给定egraph、匹配的 eclass id 和匹配生成的替换,
  // 应用重写
  fn apply_one(&self, egraph: &mut EGraph,
               id: Id, subst: &Subst) -> Vec<Id>
  {
    let (v2, e) = (subst[self.v2], subst[self.e]);
    let v2_free_in_e = egraph[e].data.free.contains(&v2);
    if v2_free_in_e {
      let mut subst = subst.clone();
      // 使用eclass id制作新的符号 (fresh symbol)
      let sym = Lambda::Symbol(format!("_{}", id).into());
      subst.insert(self.fresh, egraph.add(sym));
      // 使用修改后的 subst 应用于给定的模式
      self.if_free.apply_one(egraph, id, &subst)
    } else {
      self.if_not_free.apply_one(egraph, id, &subst)
    }
  }
}\end{lstlisting}
  % \caption{
  %   Some of the rewrites in \autoref{fig:lambda-rules} are conditional,
  %     requiring conditions like \texttt{is\_not\_same\_var} or \texttt{is\_const}.
  %   Others are fully dynamic, using a custom applier like \texttt{CaptureAvoid}
  %     instead of a syntactic right-hand side.
  %   Both conditions and custom appliers can use the computed data from the
  %     \eclass analysis; for example, \texttt{CaptureAvoid} only $\alpha$-renames if
  %     there might be a name collision.
  % }
\end{minipage}
\caption[\Eclass analysis and conditional/dynamic rewrites for the lambda calculus]{
% Our partial evaluator example highlights three important features \egg provides
%   for extensibility: \eclass analyses, conditional rewrites, and dynamic
%   rewrites.
我们的部分评估器示例突出了 \egg 提供扩展性的三个重要特性:\eclass 分析、条件重写和动态重写。
  
% The \texttt{LambdaAnalysis} type, which implements the \texttt{Analysis} trait,
%   represents the \eclass analysis.
% Its associated data (\texttt{FC}) stores
%   the constant term from that \eclass (if any) and
%   an over-approximation of the free variables used by terms in that \eclass.
% The constant term is used to perform constant folding.
% The \texttt{merge} operation implements the semilattice join, combining the free
%   variable sets and taking a constant if one exists.
% In \texttt{make}, the analysis computes the free variable sets based on the
%   \enode and the free variables of its children;
%   the \texttt{eval} generates the new constants if possible.
% The \texttt{modify} hook of \texttt{Analysis} adds the constant to the \egraph.
\texttt{LambdaAnalysis} 类型实现了 \texttt{Analysis} trait
  %\footnote{【译注】\; trait,类似抽象接口或泛型约束,是 Rust 语言中的概念}
  ,表示 \eclass 分析。 %?
它的关联数据 (\texttt{FC}) 存储来自该 \eclass 的常量项(如果有)
  和该 \eclass 中项目使用的自由变量的上近似(over-approximation)。
常量项用于进行常量折叠。
\texttt{merge} 操作实现了半格联并(semilattice join),
  结合自由变量集并采用常量(如果存在)。
在 \texttt{make} 中,分析基于 \enode 和它的子节点的自由变量集计算自由变量集;
  如果可能,\texttt{eval} 生成新的常量。
\texttt{Analysis} 的 \texttt{modify} 钩子将常量添加到 \egraph 中。


% Some of the conditional rewrites in \autoref{fig:lambda-rules} depend on
%   conditions defined here.
% Any function with the correct signature may serve as a condition.
\autoref{fig:lambda-rules} 中的一些条件重写取决于这里定义的条件。
任何具有正确签名的函数都可以作为条件。

% The \texttt{CaptureAvoid} type implements the \texttt{Applier} trait, allowing
%   it to serve as the right-hand side of a rewrite.
% \texttt{CaptureAvoid} takes two patterns and some pattern variables.
% It checks the free variable set to determine if a capture-avoiding substitution
%   is required, applying the \texttt{if\_free} pattern if so and the
%   \texttt{if\_not\_free} pattern otherwise.
\texttt{CaptureAvoid} 类型实现了 \texttt{Applier} trait,允许它作为重写的右式。
\texttt{CaptureAvoid} 接受两个模式和一些模式变量。
它检查自由变量集来确定是否需要捕获避免的替换,
  如果需要,则应用 \texttt{if\_free} 模式,
  否则应用 \texttt{if\_not\_free} 模式。
}
\label{fig:lambda-applier}
\label{fig:lambda-analysis}
\end{figure}

% 【注释翻译原文】
%   free: HashSet<Id>,    // our analysis data stores free vars
%   constant: Option<Lambda>, // and the constant value, if any

% // helper function to make pattern meta-variables

% impl Analysis<Lambda> for LambdaAnalysis {
%   type Data = FC; // attach an FC to each eclass
%   // merge implements semilattice join by joining into `to`
%   // returning true if the `to` data was modified
%   fn merge(&self, to: &mut FC, from: FC) -> bool {
%     let before_len = to.free.len();
%     // union the free variables
%     to.free.extend(from.free.iter().copied());
%     if to.constant.is_none() && from.constant.is_some() {
%       to.constant = from.constant;
%       true
%     } else {
%       before_len != to.free.len()
%     }
%   }

% // evaluate an enode if the children have constants
% // Rust's `?` extracts an Option, early returning if None

% // Functions of this type can be conditions for rewrites

% // The following two functions return closures of the
% // correct signature to be used as conditions in @\autoref{fig:lambda-rules}@.

% impl Applier<Lambda, LambdaAnalysis> for CaptureAvoid {
%   // Given the egraph, the matching eclass id, and the
%   // substitution generated by the match, apply the rewrite
%   fn apply_one(&self, egraph: &mut EGraph,
%                id: Id, subst: &Subst) -> Vec<Id>
%   {
%     let (v2, e) = (subst[self.v2], subst[self.e]);
%     let v2_free_in_e = egraph[e].data.free.contains(&v2);
%     if v2_free_in_e {
%       let mut subst = subst.clone();
%       // make a fresh symbol using the eclass id
%       let sym = Lambda::Symbol(format!("_{}", id).into());
%       subst.insert(self.fresh, egraph.add(sym));
%       // apply the given pattern with the modified subst
%       self.if_free.apply_one(egraph, id, &subst)
%     } else {
%       self.if_not_free.apply_one(egraph, id, &subst)
%     }
%   }
% }

%%% Local Variables:
%%% TeX-master: "egg"
%%% End:
