\section{\egg: 易用, 可拓展, 高效率的 \Egraphs}
% \section{\egg: Easy, Extensible, and Efficient \Egraphs}
% \section{\egg: implementation and strengths}
% 翻译完成

\label{sec:egg}
\label{sec:impl}
\label{sec:lambda}

% We implemented the techniques of rebuilding and \eclass analysis in \egg,
%   an easy-to-use, extensible, and efficient \egraph library.
% To the best of our knowledge,
%   \egg is the first general-purpose, reusable \egraph implementation.
% This has allowed focused effort on ease of use and optimization,
%   knowing that any benefits will
%   be seen across use cases as opposed to a single, ad hoc instance.
我们在 \egg 中实现了重构和 \eclass 分析技术,
  它是一个易于使用、可扩展、高效的 \egraph 库。
据我们所知,\egg 是第一个通用、可重用的 \egraph 实现。
这使得我们可以专注于易用性和优化,
  因为任何优点都将在不同的使用场景中体现,
  而不是仅限于一个特定的情况。

% This section details \egg's implementation and some of the various
%   optimizations and tools it provides to the user.
% We use an extended example of a partial evaluator for the lambda calculus\footnote{
%   \Egraphs do not have any ``built-in'' support for binding;
%   for example, equality modulo alpha renaming is not free.
%   The explicit substitution provided in this section is is illustrative but rather high in performance cost.
%   Better support for languages with binding is important future work.
% },
%   for which we provide the complete source code (which few changes for readability)
%   in \autoref{fig:lambda-lang} and \autoref{fig:lambda-analysis}.
% While contrived, this example is compact and familiar, and it highlights
%   (1) how \egg is used and (2) some of its novel features like
%   \eclass analyses and dynamic rewrites.
% It demonstrates how \egg can tackle binding,
%   a perennially tough problem for \egraphs,
%   with a simple explicit substitution approach
%   powered by \egg's extensibility.
% \autoref{sec:case-studies} goes further, providing real-world case studies of
%   published projects that have depended on \egg.
本节详细说明了 \egg 的实现和它为用户提供的各种优化和工具。
我们使用了一个对于 lambda 演算的部分求值器的扩展示例 \footnote{
  \Egraphs 没有任何“内置”的绑定支持;
  例如,“同模(equality modulo) 阿尔法重命名(alpha renaming)”不是无成本的。
  本节中提供的显式替换相当耗性能,未来的重要工作是更好的支持有绑定的语言。
},
  完整的源代码在 \autoref{fig:lambda-lang} 和 \autoref{fig:lambda-analysis} 中提供
  (仅做了少量修改以方便阅读)。
尽管示例稍显做作, 但它紧凑易懂, 并突出了
  (1) \egg 的使用方式和 (2) 其中一些新特性, 比如 \eclass 分析和动态重写。
它演示了如何借助 \egg 的可扩展性,比如使用简单的显式替换方法来解决绑定问题——
  这是一个对于\egraphs 来说永恒的难题。 
\autoref{sec:case-studies} 提供了更多真实世界的案例研究,
  展示了依赖 \egg 的已经发布的项目。

% \egg is implemented in \textasciitilde{}5000 lines of Rust,\footnote
% {
%   \citeauthor{rust} is a high-level systems programming language.
%   \egg has been integrated into applications written in other
%   programming languages using both C FFI and serialization approaches.
% }
% including code, tests, and documentation.
% \egg is open-source, well-documented, and distributed via Rust's package
%   management system.\footnote{
%   Source: \url{https://github.com/mwillsey/egg}.
%   Documentation: \url{https://docs.rs/egg}.
%   Package: \url{https://crates.io/crates/egg}.
% }
% All of \egg's components are generic over the
%   user-provided language, analysis, and cost functions.
\egg 由 \textasciitilde{}5000 行包括代码、测试和文档的 Rust \footnote
{
  \citeauthor{rust} 是一种高级系统编程语言。
  \egg 已经被集成到其他编程语言编写的应用程序中,使用 C FFI 和序列化方法。
}实现。
\egg 是开源的、文档齐全的, 通过 Rust 的包管理系统发布。\footnote{
  源码: \url{https://github.com/mwillsey/egg}.
  文档: \url{https://docs.rs/egg}.
  包: \url{https://crates.io/crates/egg}.
}
\egg 所有组件都是针对用户提供的语言、分析和成本函数的通用组件。

% \subsection{Ease of Use}
\subsection{易用}
\label{sec:egg-easy}

% 翻译完成
\begin{figure}
\begin{subfigure}[t]{0.48\linewidth}
  \begin{lstlisting}[language=Rust, basicstyle=\tiny\ttfamily, numbers=left, escapechar=|]
define_language! {
  enum Lambda {
    // 枚举变量具有数据或子元素(eclass Ids)
    // [Id; N] 是 N 个 Id 的数组

    // 基础类型操作符
    "+" = Add([Id; 2]), "=" = Eq([Id; 2]),
    "if" = If([Id; 3]),

    // 函数和绑定
    "app" = App([Id; 2]), "lam" = Lambda([Id; 2]),
    "let" = Let([Id; 3]), "fix" = Fix([Id; 2]),

    // (var x)是使用`x`作为表达式
    "var" = Use(Id),
    // (subst a x b)在 b 中替换(var x)的 a
    "subst" = Subst([Id; 3]),

    // 基础类型没有子元素,只有数据
    Bool(bool), Num(i32), Symbol(String),
  }
}

// 示例项和它们简化为的内容
// 直接从 |\egg| 测试套件中提取

test_fn! { lambda_under, rules(),
  "(lam x (+ 4 (app (lam y (var y)) 4)))"
  => "(lam x 8))",
}

test_fn! { lambda_compose_many, rules(),
  "(let compose (lam f (lam g (lam x
                (app (var f)
                     (app (var g) (var x))))))
   (let add1 (lam y (+ (var y) 1))
   (app (app (var compose) (var add1))
        (app (app (var compose) (var add1))
             (app (app (var compose) (var add1))
                  (app (app (var compose) (var add1))
                       (var add1)))))))"
  => "(lam ?x (+ (var ?x) 5))"
}

test_fn! { lambda_if_elim, rules(),
  "(if (= (var a) (var b))
       (+ (var a) (var a))
       (+ (var a) (var b)))"
  => "(+ (var a) (var b))"
}\end{lstlisting}
\end{subfigure}
\hfill
\begin{subfigure}[t]{0.48\linewidth}
  \begin{lstlisting}[language=Rust, basicstyle=\tiny\ttfamily, escapechar=|, numbers=left, firstnumber=51]
// 返回重写规则列表
fn rules() -> Vec<Rewrite<Lambda, LambdaAnalysis>> { vec![

 // open term rules 开放项规则
 rw!("if-true";  "(if  true ?then ?else)" => "?then"),
 rw!("if-false"; "(if false ?then ?else)" => "?else"),
 rw!("if-elim";  "(if (= (var ?x) ?e) ?then ?else)" => "?else"
     if ConditionEqual::parse("(let ?x ?e ?then)",
                              "(let ?x ?e ?else)")),
 rw!("add-comm";  "(+ ?a ?b)"        => "(+ ?b ?a)"),
 rw!("add-assoc"; "(+ (+ ?a ?b) ?c)" => "(+ ?a (+ ?b ?c))"),
 rw!("eq-comm";   "(= ?a ?b)"        => "(= ?b ?a)"),

 // substitution introduction 替换引入
 rw!("fix";     "(fix ?v ?e)" =>
                "(let ?v (fix ?v ?e) ?e)"),
 rw!("beta";    "(app (lam ?v ?body) ?e)" =>
                "(let ?v ?e ?body)"),

 // substitution propagation 替换传播
 rw!("let-app"; "(let ?v ?e (app ?a ?b))" =>
                "(app (let ?v ?e ?a) (let ?v ?e ?b))"),
 rw!("let-add"; "(let ?v ?e (+   ?a ?b))" =>
                "(+   (let ?v ?e ?a) (let ?v ?e ?b))"),
 rw!("let-eq";  "(let ?v ?e (=   ?a ?b))" =>
                "(=   (let ?v ?e ?a) (let ?v ?e ?b))"),
 rw!("let-if";  "(let ?v ?e (if ?cond ?then ?else))" =>
                "(if (let ?v ?e ?cond)
                     (let ?v ?e ?then)
                     (let ?v ?e ?else))"),

 // substitution elimination 替换消除
 rw!("let-const";    "(let ?v ?e ?c)" => "?c"
     if is_const(var("?c"))),
 rw!("let-var-same"; "(let ?v1 ?e (var ?v1))" => "?e"),
 rw!("let-var-diff"; "(let ?v1 ?e (var ?v2))" => "(var ?v2)"
     if is_not_same_var(var("?v1"), var("?v2"))),
 rw!("let-lam-same"; "(let ?v1 ?e (lam ?v1 ?body))" =>
                     "(lam ?v1 ?body)"),
 rw!("let-lam-diff"; "(let ?v1 ?e (lam ?v2 ?body))" =>
     ( CaptureAvoid {
        fresh: var("?fresh"), v2: var("?v2"), e: var("?e"),
        if_not_free: "(lam ?v2 (let ?v1 ?e ?body))"
                     .parse().unwrap(),
        if_free: "(lam ?fresh (let ?v1 ?e
                              (let ?v2 (var ?fresh) ?body)))"
                 .parse().unwrap(),
     })
     if is_not_same_var(var("?v1"), var("?v2"))),
]}\end{lstlisting}
\end{subfigure}
\caption[Language and rewrites for the lambda calculus in \egg]{
% \egg is generic over user-defined languages;
%   here we define a language and rewrite rules for a lambda calculus partial evaluator.
% The provided \texttt{define\_language!} macro (lines 1-22) allows the simple definition
%   of a language as a Rust \texttt{enum}, automatically deriving parsing and
%   pretty printing.
% A value of type \texttt{Lambda} is an \enode that holds either data that the
%   user can inspect or some number of \eclass children (\eclass \texttt{Id}s).
\egg 是针对用户定义语言的通用框架;
  在这里,我们为 lambda 计算的部分求值器定义了语言和重写规则。 %?lambda calculus partial evaluator
提供的 \texttt{define\_language!} 宏(行1-22)
  允许简单地将语言定义为 Rust \texttt{enum},可自动派生出解析器和产生漂亮的输出的打印器。
\texttt{Lambda} 类型的值是一个 \enode,
  它保存用户可以检查的数据或一些 \eclass 子节点(\eclass \texttt{Id}s)。

% Rewrite rules can also be defined succinctly (lines 51-100).
% Patterns are parsed as s-expressions:
%   strings from the \texttt{define\_language!} invocation (ex: \texttt{fix}, \texttt{=}, \texttt{+}) and
%   data from the variants (ex: \texttt{false}, \texttt{1}) parse as operators or terms;
%   names prefixed by ``\texttt{?}'' parse as pattern variables.
也可以简洁地定义重写规则(行51-100)。
模式被解析为 s-expressions:
  从 \texttt{define\_language!} 调用中解析的字符串
  (如: \texttt{fix}, \texttt{=}, \texttt{+})和
  从变体(variant)中解析的数据 % ? variant
  (如: \texttt{false}, \texttt{1})
  解析为运算符或项(term);% ? term
  以 ``\texttt{?}'' 为前缀的名称解析为模式变量。

% Some of the rewrites made are conditional using the
%   ``\texttt{left => right if cond}''
%   syntax.
% The \texttt{if-elim} rewrite on line 57 uses \egg's provided
%   \texttt{ConditionEqual} as a condition, only applying the right-hand side
%   if the \egraph can prove the two argument patterns equivalent.
% The final rewrite, \texttt{let-lam-diff}, is dynamic to support capture avoidance;
%   the right-hand side is a Rust value that
%   implements the \texttt{Applier} trait instead of a pattern.
% \autoref{fig:lambda-analysis} contains the supporting code for these rewrites.
其中一些重写是条件重写,使用 ``\texttt{left => right if cond}'' 语法。
在57行的 \texttt{if-elim} 重写使用 \egg 提供的
  \texttt{ConditionEqual}作为条件,
  只有在 \egraph 可以证明两个参数模式等价时才应用右式。
最终的重写 \texttt{let-lam-diff} 是动态的,用来支持捕获避免(capture avoidance);
  右边是一个实现了 \texttt{Applier} 的 trait 而不是模式的 Rust 值。% trait
\autoref{fig:lambda-analysis} 包含了这些重写的支持代码。

% We also show some of the tests (lines 27-50)
%   from \egg's \texttt{lambda} test suite.
% The tests proceed by inserting the term on the left-hand side, running
%   \egg's equality saturation, and then checking to make sure the right-hand
%   pattern can be found in the same \eclass as the initial term.
我们也展示了 \egg's \texttt{lambda} 测试套件的一些测试(行~ 27-50)。
测试的过程是在左边插入 term ,运行\egg 的 等式饱和,
  然后检查以确保右部的模式可以在与初始 term 相同的 \eclass 中找到。
}
\label{fig:lambda-rules}
\label{fig:lambda-lang}
\label{fig:lambda-examples}
\end{figure}

% 代码注释翻译:

%     // enum variants have data or children (eclass Ids)
%     // [Id; N] is an array of N `Id`s
% // 枚举变量具有数据或子元素(eclass Ids)
% // [Id; N]是 N 个 Id 的数组
%     // base type operators
% // 基础类型操作符
%     // functions and binding
% // 函数和绑定
%     // (var x) is a use of `x` as an expression
% // (var x)是使用`x`作为表达式
%     // (subst a x b) substitutes a for (var x) in b
% // (subst a x b)在 b 中替换(var x)的 a
%     // base types have no children, only data
% // 基础类型没有子元素,只有数据

% // example terms and what they simplify to
% // pulled directly from the |\egg|test suite
% 示例项和它们简化为的内容
% 直接从|\egg|测试套件中提取
% // Returns a list of rewrite rules
% 返回重写规则列表
% // open term rules
% 打开 term 规则

%%% Local Variables:
%%% TeX-master: "egg"
%%% End:


% \egg's ease of use comes primarily from its design as a library.
% By defining only a language and some rewrite rules,
%   a user can quickly
%   start developing a synthesis or optimization tool.
% Using \egg as a Rust library,
%   the user defines the language using the \texttt{define\_language!} macro
%   shown in \autoref{fig:lambda-lang}, lines 1-22.
% Childless variants in the language may contain data of user-defined types,
%   and \eclass analyses or dynamic rewrites may inspect this data.
\egg 的易用性主要来自其作为库的设计。
通过只定义一种语言和一些重写规则,用户可以快速开始开发合成或优化工具。
作为一个 Rust 库来使用 \egg ,用户可以使用 \texttt{define\_language!} 宏定义语言。
  示例见 \autoref{fig:lambda-lang} ,行1-22。
语言中的子节点为空的变量可以包含用户定义类型的数据,
  \eclass 分析或动态重写可以审查这些数据。 % ?may inspect this data.

% 原文注释:
%Defining a language is the only necessary input for \egg.
%From there, a user may create and manipulate \egraphs that hold expressions from
%  that language.
%If the user wants to perform rewrites and equality saturation, \egg provides
%  facilities for this as well.

% The user provides rewrites as shown in
%   \autoref{fig:lambda-lang}, lines 51-100.
% Each rewrite has a name, a left-hand side, and a right-hand side.
% For purely syntactic rewrites, the right-hand is simply a pattern.
% More complex rewrites can incorporate conditions or even dynamic right-hand
%   sides, both explained in the \autoref{sec:egg-extensible} and \autoref{fig:lambda-applier}.
用户可以提供重写规则,见 \autoref{fig:lambda-lang},行 51-100 。
每个重写规则都有一个名称,一个左部和一个右部。
对于纯语法重写,右部仅仅是一个模式。
更复杂的重写可以包含条件甚至是动态的右部,
  这些在 \autoref{sec:egg-extensible} 和 \autoref{fig:lambda-applier} 中有解释。

% Equality saturation workflows, regardless of the application domain,
%   typically have a similar structure:
% add expressions to an empty \egraph, run rewrites until saturation or
%   timeout, and extract the best equivalent expressions according to some cost
%   function.
% This ``outer loop'' of equality saturation involves a significant amount of
%   error-prone boilerplate:
无论应用领域如何,等式饱和工作流程通常具有相似的结构:
向空的 \egraph 添加表达式,运行重写直到饱和或超时,并根据一些代价函数提取最佳等价表达式。
这种“外循环(outer loop)”的等式饱和涉及大量的
  容易出错的繁文缛节:% ?error-prone boilerplate
% \begin{itemize}
%   \item Checking for saturation, timeouts, and \egraph size limits.
%   \item Orchestrating the read-phase, write-phase, rebuild system
%     (\autoref{fig:rebuild-code}) that makes \egg fast.
%   \item Recording performance data at each iteration.
%   \item Potentially coordinating rule execution so that expansive rules like
%     associativity do not dominate the \egraph.
%   \item Finally, extracting the best expression(s) according to a
%   user-defined cost function.
% \end{itemize}
\begin{itemize}
  \item 饱和、超时和 \egraph 大小限制的检测。
  \item 协调读取阶段、写入阶段和重建系统(\autoref{fig:rebuild-code})来加速 \egg 。
  \item 在每次迭代中记录性能数据。
  \item 潜在地协调规则的执行,
    以便像结合律(associativity)这样的扩张性(expansive)规则不会在 \egraph 中占主导地位。
  \item 最后, 根据用户定义的代价函数提取最佳表达式。
\end{itemize}

% \egg provides these functionalities through its \texttt{Runner} and
%   \texttt{Extractor} interfaces.
% \texttt{Runner}s automatically detect saturation, and can be configured to stop
%   after a time, \egraph size, or iterations limit.
% The equality saturation loop provided by \egg calls \texttt{rebuild}, so users
%   need not even know about \egg's deferred invariant maintenance.
% \texttt{Runner}s record various metrics about each iteration automatically,
%   and the user can hook into this to report relevant data.
% \texttt{Extractor}s select the optimal term from an \egraph given a
%   user-defined, local cost function.\footnote{
%     As mentioned in \autoref{sec:tricks-extraction}, extraction can be
%     implemented as part of an \eclass analysis.
%     The separate \texttt{Extractor} feature is still useful for ergonomic and
%     performance reasons.
%   }
% The two can be combined as well; users commonly record the ``best so far''
%   expression by extracting in each iteration.
\egg 通过其 \texttt{Runner} 和 \texttt{Extractor} 接口提供了这些功能。
\texttt{Runner}s 会自动检测饱和状态,并可以配置为在特定时间、\egraph 大小或迭代次数限制后停止。
由 \egg 提供的等式饱和循环会调用 \texttt{rebuild},
  因此用户甚至不需要了解 \egg 的延迟不变性(deferred invariant)维护。
\texttt{Runner}s 自动记录每次迭代的各种指标,用户可以钩入(hook)此过程以报告相关数据。
\texttt{Extractor}s 根据用户定义的局部代价函数从 \egraph 中选择最优项。\footnote{
    正如在 \autoref{sec:tricks-extraction} 中提到的,萃取可以作为 \eclass 分析的一部分实现。
    由于人性化和性能原因,独立的 \texttt{Extractor} 功能仍然有用。
  }
这两者也可以结合起来;用户通常在每次迭代中萃取来记录“到目前为止最好”的表达。

% \autoref{fig:lambda-lang} also shows \egg's \texttt{test\_fn!}
%   macro for easily creating tests (lines 27-50).
% These tests create an \egraph with the given expression, run equality saturation
%   using a \texttt{Runner}, and check to make sure the right-hand pattern can be
%   found in the same \eclass as the initial expression.
\autoref{fig:lambda-lang} 也展示了 \egg 的用于轻松创建测试的 \texttt{test\_fn!} 宏(行~27-50)。
这些测试使用给定表达式创建一个 \egraph,
  使用 \texttt{Runner} 运行 等式饱和 ,
  并检查右部的模式是否能够在与初始表达式相同的 \eclass 中找到。

% \subsection{Extensibility}
\subsection{可拓展性}
\label{sec:egg-extensible}

% For simple domains, defining a language and purely syntactic rewrites will
%   suffice.
% However, our partial evaluator requires interpreted reasoning, so we use some of
%   \egg's more advanced features like \eclass analyses and dynamic rewrites.
% Importantly, \egg supports these extensibility features as a library:
%   the user need not modify the \egraph or \egg's internals.
对于简单的领域,定义语言和纯语法重写就足够了。
但是,我们的部分评估器需要解释性推理,因此我们使用了一些 \egg 更高级的功能,如 \eclass 分析和动态重写。
重要的是,\egg 作为库支持这些可扩展性功能:用户无需修改 \egraph 或 \egg 的内部。

% 翻译完成;代码注释未翻译
\begin{figure}
\begin{minipage}[t]{0.49\linewidth}
  \begin{lstlisting}[language=Rust, basicstyle=\tiny\ttfamily, numbers=left]
type EGraph = egg::EGraph<Lambda, LambdaAnalysis>;
struct LambdaAnalysis;
struct FC {
  free: HashSet<Id>,    // 我们的分析数据存储自由变量
  constant: Option<Lambda>, // 以及常量值(如果有)
}

// 帮助函数,用于制作模式元变量(pattern meta-variables)
fn var(s: &str) -> Var { s.parse().unwrap() }

impl Analysis<Lambda> for LambdaAnalysis {
  type Data = FC; // 将 FC 附加到每个 eclass
  // merge 通过合并到 “to” 实现半格联并 (semilattice join)
  // 如果 “to” 数据被修改,则返回 true
  fn merge(&self, to: &mut FC, from: FC) -> bool {
    let before_len = to.free.len();
    // union the free variables 联并自由变量
    to.free.extend(from.free.iter().copied());
    if to.constant.is_none() && from.constant.is_some() {
      to.constant = from.constant;
      true
    } else {
      before_len != to.free.len()
    }
  }

  fn make(egraph: &EGraph, enode: &Lambda) -> FC {
    let f = |i: &Id| egraph[*i].data.free.iter().copied();
    let mut free = HashSet::default();
    match enode {
      Use(v) => { free.insert(*v); }
      Let([v, a, b]) => {
        free.extend(f(b)); free.remove(v); free.extend(f(a));
      }
      Lambda([v, b]) | Fix([v, b]) => {
        free.extend(f(b)); free.remove(v);
      }
      _ => enode.for_each_child(
             |c| free.extend(&egraph[c].data.free)),
    }
    FC { free: free, constant: eval(egraph, enode) }
  }

  fn modify(egraph: &mut EGraph, id: Id) {
    if let Some(c) = egraph[id].data.constant.clone() {
      let const_id = egraph.add(c);
      egraph.union(id, const_id);
    }
  }
}\end{lstlisting}
\end{minipage}
\hfill
\begin{minipage}[t]{0.46\linewidth}
  \begin{lstlisting}[language=Rust, basicstyle=\tiny\ttfamily, escapechar=@, numbers=left, firstnumber=51]
// 如果子元素有常量,评估 enode
// Rust的 `?` 提取一个 Option,如果是 None,则提前返回。
fn eval(eg: &EGraph, enode: &Lambda) -> Option<Lambda> {
  let c = |i: &Id| eg[*i].data.constant.clone();
  match enode {
    Num(_) | Bool(_) => Some(enode.clone()),
    Add([x, y]) => Some(Num(c(x)? + c(y)?)),
    Eq([x, y]) => Some(Bool(c(x)? == c(y)?)),
    _ => None,
  }
}

// 这种类型的函数可以作为重写的条件
trait ConditionFn = Fn(&mut EGraph, Id, &Subst) -> bool;

// 以下两个函数返回正确签名的闭包,
// 它可用作 @\autoref{fig:lambda-rules}@ 中的条件
fn is_not_same_var(v1: Var, v2: Var) -> impl ConditionFn {
    |eg, _, subst| eg.find(subst[v1]) != eg.find(subst[v2])
}
fn is_const(v: Var) -> impl ConditionFn {
     // check the LambdaAnalysis data
    |eg, _, subst| eg[subst[v]].data.constant.is_some()
}

struct CaptureAvoid {
  fresh: Var, v2: Var, e: Var,
  if_not_free: Pattern<Lambda>, if_free: Pattern<Lambda>,
}

impl Applier<Lambda, LambdaAnalysis> for CaptureAvoid {
  // 给定egraph、匹配的 eclass id 和匹配生成的替换,
  // 应用重写
  fn apply_one(&self, egraph: &mut EGraph,
               id: Id, subst: &Subst) -> Vec<Id>
  {
    let (v2, e) = (subst[self.v2], subst[self.e]);
    let v2_free_in_e = egraph[e].data.free.contains(&v2);
    if v2_free_in_e {
      let mut subst = subst.clone();
      // 使用eclass id制作新的符号 (fresh symbol)
      let sym = Lambda::Symbol(format!("_{}", id).into());
      subst.insert(self.fresh, egraph.add(sym));
      // 使用修改后的 subst 应用于给定的模式
      self.if_free.apply_one(egraph, id, &subst)
    } else {
      self.if_not_free.apply_one(egraph, id, &subst)
    }
  }
}\end{lstlisting}
  % \caption{
  %   Some of the rewrites in \autoref{fig:lambda-rules} are conditional,
  %     requiring conditions like \texttt{is\_not\_same\_var} or \texttt{is\_const}.
  %   Others are fully dynamic, using a custom applier like \texttt{CaptureAvoid}
  %     instead of a syntactic right-hand side.
  %   Both conditions and custom appliers can use the computed data from the
  %     \eclass analysis; for example, \texttt{CaptureAvoid} only $\alpha$-renames if
  %     there might be a name collision.
  % }
\end{minipage}
\caption[\Eclass analysis and conditional/dynamic rewrites for the lambda calculus]{
% Our partial evaluator example highlights three important features \egg provides
%   for extensibility: \eclass analyses, conditional rewrites, and dynamic
%   rewrites.
我们的部分评估器示例突出了 \egg 提供扩展性的三个重要特性:\eclass 分析、条件重写和动态重写。
  
% The \texttt{LambdaAnalysis} type, which implements the \texttt{Analysis} trait,
%   represents the \eclass analysis.
% Its associated data (\texttt{FC}) stores
%   the constant term from that \eclass (if any) and
%   an over-approximation of the free variables used by terms in that \eclass.
% The constant term is used to perform constant folding.
% The \texttt{merge} operation implements the semilattice join, combining the free
%   variable sets and taking a constant if one exists.
% In \texttt{make}, the analysis computes the free variable sets based on the
%   \enode and the free variables of its children;
%   the \texttt{eval} generates the new constants if possible.
% The \texttt{modify} hook of \texttt{Analysis} adds the constant to the \egraph.
\texttt{LambdaAnalysis} 类型实现了 \texttt{Analysis} trait
  %\footnote{【译注】\; trait,类似抽象接口或泛型约束,是 Rust 语言中的概念}
  ,表示 \eclass 分析。 %?
它的关联数据 (\texttt{FC}) 存储来自该 \eclass 的常量项(如果有)
  和该 \eclass 中项目使用的自由变量的上近似(over-approximation)。
常量项用于进行常量折叠。
\texttt{merge} 操作实现了半格联并(semilattice join),
  结合自由变量集并采用常量(如果存在)。
在 \texttt{make} 中,分析基于 \enode 和它的子节点的自由变量集计算自由变量集;
  如果可能,\texttt{eval} 生成新的常量。
\texttt{Analysis} 的 \texttt{modify} 钩子将常量添加到 \egraph 中。


% Some of the conditional rewrites in \autoref{fig:lambda-rules} depend on
%   conditions defined here.
% Any function with the correct signature may serve as a condition.
\autoref{fig:lambda-rules} 中的一些条件重写取决于这里定义的条件。
任何具有正确签名的函数都可以作为条件。

% The \texttt{CaptureAvoid} type implements the \texttt{Applier} trait, allowing
%   it to serve as the right-hand side of a rewrite.
% \texttt{CaptureAvoid} takes two patterns and some pattern variables.
% It checks the free variable set to determine if a capture-avoiding substitution
%   is required, applying the \texttt{if\_free} pattern if so and the
%   \texttt{if\_not\_free} pattern otherwise.
\texttt{CaptureAvoid} 类型实现了 \texttt{Applier} trait,允许它作为重写的右式。
\texttt{CaptureAvoid} 接受两个模式和一些模式变量。
它检查自由变量集来确定是否需要捕获避免的替换,
  如果需要,则应用 \texttt{if\_free} 模式,
  否则应用 \texttt{if\_not\_free} 模式。
}
\label{fig:lambda-applier}
\label{fig:lambda-analysis}
\end{figure}

% 【注释翻译原文】
%   free: HashSet<Id>,    // our analysis data stores free vars
%   constant: Option<Lambda>, // and the constant value, if any

% // helper function to make pattern meta-variables

% impl Analysis<Lambda> for LambdaAnalysis {
%   type Data = FC; // attach an FC to each eclass
%   // merge implements semilattice join by joining into `to`
%   // returning true if the `to` data was modified
%   fn merge(&self, to: &mut FC, from: FC) -> bool {
%     let before_len = to.free.len();
%     // union the free variables
%     to.free.extend(from.free.iter().copied());
%     if to.constant.is_none() && from.constant.is_some() {
%       to.constant = from.constant;
%       true
%     } else {
%       before_len != to.free.len()
%     }
%   }

% // evaluate an enode if the children have constants
% // Rust's `?` extracts an Option, early returning if None

% // Functions of this type can be conditions for rewrites

% // The following two functions return closures of the
% // correct signature to be used as conditions in @\autoref{fig:lambda-rules}@.

% impl Applier<Lambda, LambdaAnalysis> for CaptureAvoid {
%   // Given the egraph, the matching eclass id, and the
%   // substitution generated by the match, apply the rewrite
%   fn apply_one(&self, egraph: &mut EGraph,
%                id: Id, subst: &Subst) -> Vec<Id>
%   {
%     let (v2, e) = (subst[self.v2], subst[self.e]);
%     let v2_free_in_e = egraph[e].data.free.contains(&v2);
%     if v2_free_in_e {
%       let mut subst = subst.clone();
%       // make a fresh symbol using the eclass id
%       let sym = Lambda::Symbol(format!("_{}", id).into());
%       subst.insert(self.fresh, egraph.add(sym));
%       // apply the given pattern with the modified subst
%       self.if_free.apply_one(egraph, id, &subst)
%     } else {
%       self.if_not_free.apply_one(egraph, id, &subst)
%     }
%   }
% }

%%% Local Variables:
%%% TeX-master: "egg"
%%% End:


% \autoref{fig:lambda-applier} shows the remainder of the code for our lambda
%   calculus partial evaluator.
% It uses an \eclass analysis (\texttt{LambdaAnalysis})
%   to track free variables and constants associated
%   with each \eclass.
% The implementation of the \eclass analysis is in Lines 11-50.
% The \eclass analysis invariant
%   guarantees that the analysis data contains an over-approximation of free variables
%   from terms represented in that \eclass.
% The analysis also does constant folding
%   (see the \texttt{make} and \texttt{modify} methods).
\autoref{fig:lambda-applier} 展示了我们的 lambda 部分求值器的代码的剩余部分。
它使用一个 \eclass 分析 (\texttt{LambdaAnalysis})来跟踪与每个 \eclass 相关联的自由变量和常量。
\eclass 分析的实现在第 11-50 行。
\eclass 分析不变量保证了
  分析数据包含来自该 \eclass 表示的 term 的
  自由变量的上近似(或作“过近似”,Over-approximation)。%?
该分析还进行了常量折叠(请参阅 \texttt{make} 和 \texttt{modify} 方法)。
% The \texttt{let-lam-diff} rewrite (Line 90, \autoref{fig:lambda-rules})
%   uses the \texttt{CaptureAvoid} (Lines 81-100, \autoref{fig:lambda-applier})
%   dynamic right-hand side to do capture-avoiding
%   substitution only when necessary based on the free variable information.
% The conditional rewrites from \autoref{fig:lambda-rules} depend on the
%   conditions \texttt{is\_not\_same\_var} and
%   \texttt{is\_var} (Lines 68-74, \autoref{fig:lambda-applier})
%   to ensure correct substitution.
\texttt{let-lam-diff} 重写(\autoref{fig:lambda-rules},第 90 行)
  使用 \texttt{CaptureAvoid} (\autoref{fig:lambda-applier},第 81-100 行)
  的动态右部,根据自由变量的信息仅在必要时进行捕获避免地替换(capture-avoiding substitution)。
\autoref{fig:lambda-rules} 的条件重写取决于条件 \texttt{is\_not\_same\_var} 和 \texttt{is\_var} 
  ( \autoref{fig:lambda-applier},第 68-74 行) 以确保正确的替换。

% \egg is extensible in other ways as well.
% As mentioned above, \texttt{Extractor}s are parameterized by a user-provided
%   cost function.
% \texttt{Runner}s are also extensible with user-provided rule schedulers that can
%   control the behavior of potentially troublesome rewrites.
% \label{sec:rule-scheduling}
% In typical equality saturation, each rewrite is searched for and applied each
%   iteration.
% This can cause certain rewrites, commonly associativity or distributivity,
%   to dominate others and make the search space less productive.
% Applied in moderation, these rewrites can trigger other rewrites and find
%   greatly improved expressions,
%   but they can also slow the search by
%   exploding the \egraph exponentially in size.
% By default, \egg uses the built-in backoff scheduler
%   that identifies rewrites that are matching in exponentially-growing
%   locations and temporarily bans them.
% We have observed that this greatly reduced run time (producing the same results)
%   in many settings.
% \egg can also use a conventional every-rule-every-time scheduler, or the user
%   can supply their own.
\egg 在其他方面也是可扩展的。
如上所述,\texttt{Extractor} 由用户提供的成本函数来参数化。% are parameterized
\texttt{Runner} 也可以使用用户提供的规则调度程序进行扩展,以控制潜在有问题的重写的行为。
\label{sec:rule-scheduling}
在典型的等式饱和中,每次迭代都会搜索和应用每个重写。
这可能导致某些重写(通常是结合率或分配率)占据其他重写的地位,使搜索空间变得不够高效。
适量使用这些重写可以触发其他重写并找到更好的表达式,
  但它们也可能会使搜索变慢,因为它们会使 \egraph 的大小指数级增长。
默认情况下,\egg 使用内置的退避调度程序(backoff scheduler),
  该调度程序识别在指数级位置匹配的重写并暂时禁用它们。
我们已经观察到,这在许多情况下大大减少了运行时间(产生相同的结果)。
\egg 也可以使用常规的每规则每次调度器(every-rule-every-time scheduler),
  或者用户自定义的调度器。

% \subsection{Efficiency}
\subsection{效率}
\label{sec:egg-efficient}

% \egg's novel \textit{rebuilding} algorithm (\autoref{sec:rebuild})
% combined with systems programming best practices
%   makes \egraphs---and the equality saturation
%   use case in particular---more efficient than prior tools.
\egg 的新颖的\textit{重建}算法 (\autoref{sec:rebuild}) 与系统编程最佳实践相结合,
  使得 \egraphs —— 尤其是等式饱和使用案例 —— 比之前的工具更有效率。

% \egg is implemented in Rust, giving the compiler freedom to
%   specialize and inline user-written code.
% This is especially important as
%   \egg's generic nature leads to tight interaction
%   between library code
%   (e.g., searching for rewrites) and user code (e.g., comparing operators).
% \egg is designed from the ground up to use cache-friendly,
%   flat buffers with minimal indirection for most internal data structures.
% This is in sharp contrast to traditional representations of \egraphs
%   \cite{nelson, simplify} that contains many tree- and linked list-like data
%   structures.
% \egg additionally compiles patterns to be executed by a small virtual machine
%   \cite{ematching}, as opposed to recursively walking the tree-like
%   representation of patterns.
\egg 是用 Rust 实现的,这使得编译器可以自由地特化(specialize)和内联(inline)用户编写的代码。
这非常重要,因为 \egg 的通用性导致了库代码(例如搜索重写)和用户代码(例如比较运算符)之间的紧密交互。
\egg 从头开始设计,
  使用缓存友好的、
  带有最少间接层(indirection)的平面缓冲区(flat buffers),%?flat buffers、indirection
  用于大多数内部数据结构。
这与包含许多树和链表类似的数据结构的传统 \egraphs 表示形式 \cite{nelson, simplify} 形成鲜明对比。
与递归遍历模式的树状表示法相比,\egg 另外还编译了模式,% ?additionally compiles patterns
  以便由一个小型虚拟机执行 \cite{ematching}。

% Aside from deferred rebuilding, \egg's equality saturation algorithm leads to
%   implementation-level performance enhancements.
% Searching for rewrite matches, which is the bulk of running time, can be
%   parallelized thanks to the phase separation.
% Either the rules or \eclasses could be searched in parallel.
% Furthermore, the once-per-iteration frequency of rebuilding allows \egg to
%   establish other performance-enhancing invariants that hold during the
%   read-only search phase.
% For example, \egg sorts \enodes within each \eclass to enable binary search, and
%   also maintains a cache mapping function symbols to \eclasses that
%   contain \enodes with that function symbol.
除了延迟重建外,\egg 的等式饱和算法还带来了实现层面的性能增强。
搜索重写匹配,这是运行时间的主要部分,可以通过阶段分离并行化。
规则或 \eclasses 可以并行搜索。
此外,每次迭代重建的频率允许 \egg 建立其他在只读搜索阶段期间保持的性能增强不变量,。
此外,每次迭代一次(once-per-iteration)的重建频率
  允许 \egg 建立其他增强性能的不变量,
  这些不变量在只读搜索阶段保持不变。
例如,\egg 在每个 \eclass 内排序 \enodes 以启用二分查找,
  并维护将函数符号映射到包含具有该函数符号的 \enodes 的 \eclasses 的缓存。

% Many of \egg's extensibility features can also be used to improve performance.
% As mentioned above, rule scheduling can lead to great performance improvement in
%   the face of ``expansive'' rules that would otherwise dominate the search
%   space.
% The \texttt{Runner} interface also supports user hooks that can stop
%   the equality saturation after some arbitrary condition.
% This can be useful when using equality saturation to prove terms equal; once
%   they are unified, there is no point in continuing.
% \label{sec:egg-batched}
% \egg's \texttt{Runner}s also support batch simplification, where multiple terms
%   can be added to the initial \egraph before running equality saturation.
% If the terms are substantially similar, both rewriting and any \eclass analyses
%   will benefit from the \egraph's inherent structural deduplication.
% The case study in \autoref{sec:herbie} uses batch simplification to achieve
%   a large speedup with simplifying similar expressions.
\egg 的许多可扩展性功能也可用于提高性能。
如上所述,规则调度可以在面对“扩展性”规则时带来巨大的性能改进,否则这些规则将主导搜索空间。
\texttt{Runner} 接口也支持用户钩子,可以在任意条件后停止equality saturation。
当使用等式饱和来证明项目相等时,这是非常有用的; 
  一旦它们一致,就没有继续的必要了。
\label{sec:egg-batched}
\egg 的 \texttt{Runner} 也支持批量简化,
  在运行等式饱和之前可以将多个 term 添加到初始 \egraph。
如果这些 term 显着相似,重写和任何 \eclass 分析都将从 \egraph 的固有结构去重复中受益。
\autoref{sec:herbie} 中的案例研究使用批量简化来实现简化相似表达式的大幅加速。

%%% Local Variables:
%%% TeX-master: "egg"
%%% End:
