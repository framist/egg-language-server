% 翻译完成
\begin{figure}
\begin{subfigure}[t]{0.48\linewidth}
  \begin{lstlisting}[language=Rust, basicstyle=\tiny\ttfamily, numbers=left, escapechar=|]
define_language! {
  enum Lambda {
    // 枚举变量具有数据或子元素(eclass Ids)
    // [Id; N] 是 N 个 Id 的数组

    // 基础类型操作符
    "+" = Add([Id; 2]), "=" = Eq([Id; 2]),
    "if" = If([Id; 3]),

    // 函数和绑定
    "app" = App([Id; 2]), "lam" = Lambda([Id; 2]),
    "let" = Let([Id; 3]), "fix" = Fix([Id; 2]),

    // (var x)是使用`x`作为表达式
    "var" = Use(Id),
    // (subst a x b)在 b 中替换(var x)的 a
    "subst" = Subst([Id; 3]),

    // 基础类型没有子元素,只有数据
    Bool(bool), Num(i32), Symbol(String),
  }
}

// 示例项和它们简化为的内容
// 直接从 |\egg| 测试套件中提取

test_fn! { lambda_under, rules(),
  "(lam x (+ 4 (app (lam y (var y)) 4)))"
  => "(lam x 8))",
}

test_fn! { lambda_compose_many, rules(),
  "(let compose (lam f (lam g (lam x
                (app (var f)
                     (app (var g) (var x))))))
   (let add1 (lam y (+ (var y) 1))
   (app (app (var compose) (var add1))
        (app (app (var compose) (var add1))
             (app (app (var compose) (var add1))
                  (app (app (var compose) (var add1))
                       (var add1)))))))"
  => "(lam ?x (+ (var ?x) 5))"
}

test_fn! { lambda_if_elim, rules(),
  "(if (= (var a) (var b))
       (+ (var a) (var a))
       (+ (var a) (var b)))"
  => "(+ (var a) (var b))"
}\end{lstlisting}
\end{subfigure}
\hfill
\begin{subfigure}[t]{0.48\linewidth}
  \begin{lstlisting}[language=Rust, basicstyle=\tiny\ttfamily, escapechar=|, numbers=left, firstnumber=51]
// 返回重写规则列表
fn rules() -> Vec<Rewrite<Lambda, LambdaAnalysis>> { vec![

 // open term rules 开放项规则
 rw!("if-true";  "(if  true ?then ?else)" => "?then"),
 rw!("if-false"; "(if false ?then ?else)" => "?else"),
 rw!("if-elim";  "(if (= (var ?x) ?e) ?then ?else)" => "?else"
     if ConditionEqual::parse("(let ?x ?e ?then)",
                              "(let ?x ?e ?else)")),
 rw!("add-comm";  "(+ ?a ?b)"        => "(+ ?b ?a)"),
 rw!("add-assoc"; "(+ (+ ?a ?b) ?c)" => "(+ ?a (+ ?b ?c))"),
 rw!("eq-comm";   "(= ?a ?b)"        => "(= ?b ?a)"),

 // substitution introduction 替换引入
 rw!("fix";     "(fix ?v ?e)" =>
                "(let ?v (fix ?v ?e) ?e)"),
 rw!("beta";    "(app (lam ?v ?body) ?e)" =>
                "(let ?v ?e ?body)"),

 // substitution propagation 替换传播
 rw!("let-app"; "(let ?v ?e (app ?a ?b))" =>
                "(app (let ?v ?e ?a) (let ?v ?e ?b))"),
 rw!("let-add"; "(let ?v ?e (+   ?a ?b))" =>
                "(+   (let ?v ?e ?a) (let ?v ?e ?b))"),
 rw!("let-eq";  "(let ?v ?e (=   ?a ?b))" =>
                "(=   (let ?v ?e ?a) (let ?v ?e ?b))"),
 rw!("let-if";  "(let ?v ?e (if ?cond ?then ?else))" =>
                "(if (let ?v ?e ?cond)
                     (let ?v ?e ?then)
                     (let ?v ?e ?else))"),

 // substitution elimination 替换消除
 rw!("let-const";    "(let ?v ?e ?c)" => "?c"
     if is_const(var("?c"))),
 rw!("let-var-same"; "(let ?v1 ?e (var ?v1))" => "?e"),
 rw!("let-var-diff"; "(let ?v1 ?e (var ?v2))" => "(var ?v2)"
     if is_not_same_var(var("?v1"), var("?v2"))),
 rw!("let-lam-same"; "(let ?v1 ?e (lam ?v1 ?body))" =>
                     "(lam ?v1 ?body)"),
 rw!("let-lam-diff"; "(let ?v1 ?e (lam ?v2 ?body))" =>
     ( CaptureAvoid {
        fresh: var("?fresh"), v2: var("?v2"), e: var("?e"),
        if_not_free: "(lam ?v2 (let ?v1 ?e ?body))"
                     .parse().unwrap(),
        if_free: "(lam ?fresh (let ?v1 ?e
                              (let ?v2 (var ?fresh) ?body)))"
                 .parse().unwrap(),
     })
     if is_not_same_var(var("?v1"), var("?v2"))),
]}\end{lstlisting}
\end{subfigure}
\caption[Language and rewrites for the lambda calculus in \egg]{
% \egg is generic over user-defined languages;
%   here we define a language and rewrite rules for a lambda calculus partial evaluator.
% The provided \texttt{define\_language!} macro (lines 1-22) allows the simple definition
%   of a language as a Rust \texttt{enum}, automatically deriving parsing and
%   pretty printing.
% A value of type \texttt{Lambda} is an \enode that holds either data that the
%   user can inspect or some number of \eclass children (\eclass \texttt{Id}s).
\egg 是针对用户定义语言的通用框架;
  在这里,我们为 lambda 计算的部分求值器定义了语言和重写规则。 %?lambda calculus partial evaluator
提供的 \texttt{define\_language!} 宏(行1-22)
  允许简单地将语言定义为 Rust \texttt{enum},可自动派生出解析器和产生漂亮的输出的打印器。
\texttt{Lambda} 类型的值是一个 \enode,
  它保存用户可以检查的数据或一些 \eclass 子节点(\eclass \texttt{Id}s)。

% Rewrite rules can also be defined succinctly (lines 51-100).
% Patterns are parsed as s-expressions:
%   strings from the \texttt{define\_language!} invocation (ex: \texttt{fix}, \texttt{=}, \texttt{+}) and
%   data from the variants (ex: \texttt{false}, \texttt{1}) parse as operators or terms;
%   names prefixed by ``\texttt{?}'' parse as pattern variables.
也可以简洁地定义重写规则(行51-100)。
模式被解析为 s-expressions:
  从 \texttt{define\_language!} 调用中解析的字符串
  (如: \texttt{fix}, \texttt{=}, \texttt{+})和
  从变体(variant)中解析的数据 % ? variant
  (如: \texttt{false}, \texttt{1})
  解析为运算符或项(term);% ? term
  以 ``\texttt{?}'' 为前缀的名称解析为模式变量。

% Some of the rewrites made are conditional using the
%   ``\texttt{left => right if cond}''
%   syntax.
% The \texttt{if-elim} rewrite on line 57 uses \egg's provided
%   \texttt{ConditionEqual} as a condition, only applying the right-hand side
%   if the \egraph can prove the two argument patterns equivalent.
% The final rewrite, \texttt{let-lam-diff}, is dynamic to support capture avoidance;
%   the right-hand side is a Rust value that
%   implements the \texttt{Applier} trait instead of a pattern.
% \autoref{fig:lambda-analysis} contains the supporting code for these rewrites.
其中一些重写是条件重写,使用 ``\texttt{left => right if cond}'' 语法。
在57行的 \texttt{if-elim} 重写使用 \egg 提供的
  \texttt{ConditionEqual}作为条件,
  只有在 \egraph 可以证明两个参数模式等价时才应用右式。
最终的重写 \texttt{let-lam-diff} 是动态的,用来支持捕获避免(capture avoidance);
  右边是一个实现了 \texttt{Applier} 的 trait 而不是模式的 Rust 值。% trait
\autoref{fig:lambda-analysis} 包含了这些重写的支持代码。

% We also show some of the tests (lines 27-50)
%   from \egg's \texttt{lambda} test suite.
% The tests proceed by inserting the term on the left-hand side, running
%   \egg's equality saturation, and then checking to make sure the right-hand
%   pattern can be found in the same \eclass as the initial term.
我们也展示了 \egg's \texttt{lambda} 测试套件的一些测试(行~ 27-50)。
测试的过程是在左边插入 term ,运行\egg 的 等式饱和,
  然后检查以确保右部的模式可以在与初始 term 相同的 \eclass 中找到。
}
\label{fig:lambda-rules}
\label{fig:lambda-lang}
\label{fig:lambda-examples}
\end{figure}

% 代码注释翻译:

%     // enum variants have data or children (eclass Ids)
%     // [Id; N] is an array of N `Id`s
% // 枚举变量具有数据或子元素(eclass Ids)
% // [Id; N]是 N 个 Id 的数组
%     // base type operators
% // 基础类型操作符
%     // functions and binding
% // 函数和绑定
%     // (var x) is a use of `x` as an expression
% // (var x)是使用`x`作为表达式
%     // (subst a x b) substitutes a for (var x) in b
% // (subst a x b)在 b 中替换(var x)的 a
%     // base types have no children, only data
% // 基础类型没有子元素,只有数据

% // example terms and what they simplify to
% // pulled directly from the |\egg|test suite
% 示例项和它们简化为的内容
% 直接从|\egg|测试套件中提取
% // Returns a list of rewrite rules
% 返回重写规则列表
% // open term rules
% 打开 term 规则

%%% Local Variables:
%%% TeX-master: "egg"
%%% End:
